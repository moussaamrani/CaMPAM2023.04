\subsection{Conditions}
\label{sec:Conditions}

\textsf{Condition}s can enrich a \textsf{Suggestion} with domain knowledge to 
specify constraints about the context, in which the \textsf{Suggestion}s 
\textsf{Recommendation} is valid. A \textsf{Condition} can specify constraints 
about elements of the \metamodel, its instance, the \viewtype \metamodel, and 
its instance. 
Since this part is still ongoing work, it is presented as an abstract class 
in \cref{fig:Suggestion}.

A \textsf{Condition} for our example could be the \viewtype that 
only displays the basic \textsf{FSM} depicted in \cref{fig:FSM:Init}. 
The \textsf{Recommendation} for the \textsf{Suggestion} to the refactoring that 
adds guards would be to not include the \metamodel changes into the \viewtype as
it only displays the basic \textsf{FSM}. \textsf{Conditions} may also form a 
domain specific hierarchy for the refinement of \textsf{Suggestion}s. A refined 
\textsf{Condition} could include the basic \textsf{FSM} \viewtype and additionally
the condition, that it is a legacy \viewtype. Because legacy \viewtypes are used
by other tools, they cannot be modified. With this additional constraint, the 
\textsf{Recommendation} and thus the \textsf{Suggestion} can be refined to not 
include any \metamodel change into the \viewtype as it would break its legacy 
property.