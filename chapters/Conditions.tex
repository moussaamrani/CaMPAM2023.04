\subsection{Condition}
\label{sec:Conditions}
\textsf{Condition}s can enrich a \textsf{Suggestion} with domain knowledge to 
specify constraints about the context, in which the \textsf{Suggestion}'s 
\textsf{Recommendation} is valid. A \textsf{Condition} can specify constraints 
about elements of the \metamodel, its instances in a model, the \viewtype 
\metamodel, and its instances (i.e. the views). 
Since this part is still ongoing work, it is presented as an abstract class 
in \cref{fig:Suggestion}.

A \textsf{Condition} for our example could be the \viewtype that 
only displays the basic \textsf{FSM} depicted in \cref{fig:FSM:Init}. 
The \textsf{Recommendation} for the \textsf{Suggestion} to the refactoring that 
adds guards would be to not include the \metamodel changes into the \viewtype if the latter's purpose is only to display the basic \textsf{FSM}.
% \LC{Thomas, I rephrased preceding sentence. Please see if you agree?}\TW{I agree} 
\textsf{Conditions} may also form a 
domain specific hierarchy for the refinement of \textsf{Suggestion}s. A refined 
\textsf{Condition} could include the basic \textsf{FSM} \viewtype and additionally
the condition, that it is a legacy \viewtype, i.e., that there are external dependencies on it: %\LC{The meaning of 'legacy \viewtype' needs introduction here or beforehand.}\LK{We could add something like ``, i.e., that there are external dependencies on it'', but isn't that pretty much the same as the following sentence?}
they are used by other tools and hence cannot just be modified. With this additional constraint, the 
\textsf{Recommendation} and thus the \textsf{Suggestion} can be refined to not 
include any \metamodel change into the \viewtype as it would break its legacy 
property.