\subsection{Conditions}
\label{sec:Conditions}
\textsf{Condition}s can enrich a \textsf{Suggestion} with domain knowledge to specify constraints about the context, in which the \textsf{Suggestion}s \textsf{Recommendation} is valid. A \textsf{Condition} can specify constraints about elements of the \metamodel, its instance, the \viewtype \metamodel, and its instance. A \textsf{Condidtion} for our example could be the \viewtype that only displays the basic \textsf{FSM} depicted in \cref{fig:FSM:Init}. The \textsf{Recommendation} for the \textsf{Suggestion} to the refactoring that adds guards would be to not include the \metamodel changes into the \viewtype as it only displays the basic \textsf{FSM}. \textsf{Conditions} may also form a domain specific hierarchy for the refinement of \textsf{Suggestion}s. A refined \textsf{Condition} could include the basic \textsf{FSM} \viewtype and additionally the condition, that it is a legacy \viewtype. Because legacy \viewtypes are used by other tools, they cannot be modified. With this additional constraint, the \textsf{Recommendation} and thus the \textsf{Suggestion} can be refined to not include any \metamodel change into the \viewtype as it would break its legacy property.

% some Conditions providing additional
% information on the applicability of Operators, which can be
% extended or enriched with domain knowledge for generating
% more specific suggestions. Moussa ▶For now, I still don’t
% see where these would be helpful in our example. We may still
% discuss that in general in the Discussion Section, but in that
% case, we should remove it from the Suggestion metamodel!◀
% \LC{todo?}
% \TW{The constraints are defined on either the \metamodel, its instance, the view generation operators or the target \viewtype \metamodel. Constraints may also form a hierarchy for the refinement of suggestions. The $SUC$ is either a textual description of the suggestion or a formalisation of possible changes or a change on the \viewtype \metamodel or a combination of all three. Either the textual description or the formalisation and the possible changes are preferable as it enables the methodologist to precisely define the change on the \viewtype \metamodel as well as giving some rationale. The formalization represents the middle way between the directly applicable definition of the changes and the textual description.}