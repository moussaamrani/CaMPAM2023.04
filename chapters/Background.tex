\section{Background}
\label{sec:Background}

%\LC{General style thing---in good old Strunk and White's style guide, and in most texts I've read, style is to use initial capital when using a section/figure/table/... identifier, no initial capital when not. So: Section~1, the next section, Table~2, the preceding table, etc. We're currently not consistent in that.}
%\LK{So the solution should be to use cref, right? As far as I can see, this is done everywhere in this section already, or did I miss something?}
%\LC{No, it does not seem to solve this? E.g.~right below these comments, there's a 'section II-A' currently---not 'Section II-A'}
%\LK[Yes, what I meant is you can simply set the `capitalise` option for the cleveref package.]

In this section, we explain the necessary concepts on which our approach builds.
First, in~\cref{sec:ViewBasedDevelopment}, we cover different solution strategies to view-based development, and briefly introduce the development process in view-based environments.
\Cref{sec:ViewGeneration} explains how views are generated, and the impact of view generation languages on our approach.
Finally, in~\cref{sec:MetaModelEvolution}, we explain how we break down \metamodel evolution in semantically meaningful steps and how these steps can be detected a posteriori.

% \MA{Does it make sense to split the Section into two: (i) Views; (ii) View Generation (or something similar)}

\subsection{View-Based Development}
\label{sec:ViewBasedDevelopment}

%\LC{This subsection imho could use shortening but more importantly, I think we should link it to our context---is it the case that we consider a context with projective approaches based on pragmatic SUMs? It seems to me so, but we do not state this. Perhaps start this subsection with that information '... a so-called projective approach based on a so-called pragmatic SUMS. With respect to the projective approach, ...' ?}

The main challenges for view-based development is keeping the views on the models of the system consistent.
For our approach, we will focus on \emph{projective} approaches for this.
In contrast to \emph{synthetic} approaches, where direct connections between the views are used to keep them consistent, projective approaches use an additional underlying model, which is the source of all presented information \autocite{atkinson_fundamental_2015}.
Views in projective approaches are transient, i.e., not persistent, and generated dynamically from the underlying model whenever they are required \autocite{atkinson_orthographic_2010}.
With projective views, the user is only allowed to interact with the system through the views, while the underlying model is hidden.
In particular, when a user wants to make changes to the system, the changes need to be applied to (editable) views first and are then propagated to the underlying model.
The relationship between views, \viewtypes, models and \metamodels in projective approaches is shown in~\cref{fig:view_concept}.
The ISO-42010 standard \autocite{ISO42010} also contains a definition of the \emph{view} concept, as well as of synthetic and projective approaches, but in a broader sense for software architecture and software architecture description languages.

\begin{comment}
There exist \emph{synthetic} and \emph{projective} approaches to view-based modeling \autocite{atkinson_fundamental_2015}.
In the former approaches, the system is represented by the union of all views.
For that purpose, changes must be propagated between the views to ensure the system description is consistent.
This is different in projective approaches where, in addition to the views, an underlying model of the system exists, which is the source of all information.    
\end{comment}

\begin{figure}
    \begin{center}
        \includegraphics{images/view-terminology.tikz}
    \end{center}

    \caption{Visualization of the view terminology for projective approaches \autocite{klare_enabling_2021}.}
    \label{fig:view_concept}
\end{figure}

For projective view-based environments there are different ways of constructing the underlying model.
An important differentiation is between \emph{essential} and \emph{pragmatic} construction approaches \autocite{atkinson_fundamental_2015}.
With an essential SUM (Single Underlying Model) there is a single model, which contains the complete system description and is free of any internal redundancies \autocite{atkinson_orthographic_2010}.
While it may be difficult to construct a meta-model for this, especially in domain-spanning system development, synchronisation between the views is reduced to writing any changes made on them back to the single underlying model.
This is different for pragmatic approaches where the underlying model consists of multiple submodels, which allows re-using existing meta-models, e.g., from model-based tools, but requires explicit consistency specifications between the underlying meta-models \autocite{klare_enabling_2021}.
While our approach considers only view types defined on a single meta-model, it is agnostic to the construction of the underlying model and could be extended to view types depending on multiple meta-models in the future to fully support the pragmatic approach as well.

\begin{comment}
\Textcite{atkinson_fundamental_2015} differentiate between \emph{essential} and \emph{pragmatic} SUMs.
With an essential SUM there is a single model, which contains the complete system description and is free of any internal redundancies.
For the synchronization between the views, this is the easiest solution, because the views are simply required to write any changes made on them back to the SUM.
% \LC{the? 'the changes made in them [i.e. the views], I reckon?}
Subsequently generated views are then automatically consistent.
An example of an approach employing an essential SUM is OSM by \textcite{atkinson_orthographic_2010}.
In contrast to a single redundancy-free model, there are also pragmatic approaches where the underlying model consists of multiple submodels.
The main benefit of pragmatic approaches is the construction of the underlying model.
While it is difficult to create a single redundancy-free \metamodel, especially for a domain-spanning system, pragmatic SUMs allow the integration of already available \metamodels, e.g., from development tools used in the various domains.
Additional effort is, however, necessary to keep the models consistent.
One framework for constructing pragmatic SUMs, here called virtual SUMs (V-SUMs), is Vitruvius \autocite{klare_enabling_2021}.
\end{comment}

For the process of constructing the view-based development environment, \textcite{atkinson_orthographic_2010} identified two involved roles.
The first role, the \emph{methodologist}, is responsible for creating the view-based environment, including the underlying model and the view types.
In addition, they define consistency specifications for the underlying model, if necessary, and rules for how changes on views are propagated back to the underlying model.
The second role, the \emph{developer}, uses the environment created by the methodologist to develop the actual system.
For this, the developer creates views on the system to inspect its properties and performs changes on them, which are then applied to the underlying model.

\subsection{View Generation}
\label{sec:ViewGeneration}

Since the content of projective views is derived from the underlying model, the methodologist must define a \emph{view generation transformation (VGT)} for each view type, which dynamically creates a view from the model instances when required \autocite{tunjic_synchronization_2015}.
% For pragmatic SUMs, the VGTs must be able to combine models from multiple \metamodels, since the content of a view can be scattered over different internal \metamodels.
%\LC{but we do not consider such a scenario in this paper; remove sentence? Perhaps move reference to Section 1 where this is mentioned.}
For users to make changes to the system, views in general must be editable, which requires at least partially bidirectional VGTs.
This may either be achieved by having a transformation language that can derive the reverse of a specified view generation transformation or by manually writing transformations for both directions, i.e., the view generation transformation and the transformation of view-level changes to underlying model-level changes.
% \MA{The term "bidirectional" assumes that you only define one direction, and the Trafo Framework automatically infers the other direction. This is usually too much trouble: having a spec of both sense can do the trick.}
Of course, backpropagation of changes is not always possible, e.g., when a view shows a sum of multiple values, it has to be read-only in the view.
Furthermore, the methodologist might restrict such backpropagation further by disallowing the editing of elements that should not be edited, e.g., when the view is intended for a specific task.
VGTs can be defined using common model transformation languages, as well as general purpose programming languages.
For our approach, we do not assume a specific language to be used, but we require that the elements of the view types can be traced back to the meta-model elements they represent.

\begin{comment}
For the definition of the VGTs, common model transformation languages, such as QVT \cite{omg_qvt} or ATL \cite{eclipse_atl}, can be used, as well as formal bidirectional model transformation frameworks, such as the \emph{lenses} framework \autocite{foster_combinators_2007}.
% \MA{I would have expected a short discussion on OCL beyond the obvious limitation (of handling only one MM).}
The query part of the constraint language OCL \cite{omg_ocl} can also be used for the generation of simple views.
However, because it does not support deriving new model elements from existing ones and is limited to a single source \metamodel, it is not sufficient for our envisioned scenario.
Another approach for the definition of VGTs is \emph{ModelJoin} \autocite{burger_model-join_2016}, which is an operator-based model query language.
ModelJoin also supports the derivation of a view type based on a query.
%\LC{Missing a conclusion here/why the paragaph is here. What are we assuming in our approach / our example scenario, any specific approach for VGT definition? Do we need to discuss all the options in this paragraph?}

In our work, we assume that a model query language similar to ModelJoin is used to define the VGTs, since we require semantic information on how a model element is used in a view type.
Our approach will derive this information from the query language operators in which model elements are used in a VGT.
The operators, which we assume to be used in model query languages for view generation, are shown in~\cref{fig:model-query-operators}.

\begin{figure}[h]
    \begin{itemize}
        \item \texttt{Select}
        \item \texttt{Filter}
        \item \texttt{Join}
        \item \texttt{Aggregate}
        \item \texttt{Calculate}
    \end{itemize}

    \caption{Proposed set of operators for a model query language for view generation based on \textcite{burger_model-join_2016}.}
    \label{fig:model-query-operators}
\end{figure}

The expected behavior of the model query operators in~\cref{fig:model-query-operators} is as follows.
The \texttt{select} operator selects \metamodel elements, which will appear in the view type, while the \texttt{filter} operator is used to define criteria by which model instances are included in a view when generating it.
\MA{So, if I understand correctly, a "select" is a "filter" with no filtering criterium?}
\LK{Not directly. Comparing it to databases, the "select" operator would create the column headers, i.e. specify what kind of data will be presented, while the "filter" operator would define (e.g., by defining a predicate function) which rows from the database will be included in the result. Is this understandable from what I wrote or do you have an idea how I could improve the explanation?}
For the multi-\metamodel case, the \texttt{join} operator connects model instances from different \metamodels.
Both, the \texttt{aggregate} and the \texttt{calculate} operator create new model elements.
While the former combines the values of a single property in multiple instances of a \metaclass, the latter is used to calculate a new property based on multiple existing ones.
Following \textcite{burger_model-join_2016}, we assume this is a reasonable set of operators for the definition of generic views.
\end{comment}

\subsection{\Metamodel Evolution}
\label{sec:MetaModelEvolution}

In order to reason about \metamodel evolution, a description of the difference between a \metamodel before and after an evolution process is required.
For analyzing this difference, a common representation of the difference is a list of smaller changes or evolution steps.
For us to provide meaningful suggestions or even automate certain co-evolution steps, the description of the evolution steps should contain semantic information about the kind of change that is represented by it.
%\LC{Next two sentences need improving. 'On the other hand' but no 'On the one hand' is odd. Also, the word 'meaningful' carries a lot of weight here. Do you mean to say that add, delete, update are not meaningful?}
However, defining a set of semantically meaningful evolution steps, which at the same time is complete in the sense that every difference between two \metamodels can be constructed from it, is neither realistic nor necessarily useful.
As a solution for this, we distinguish between \emph{atomic} \metamodel change types, which jointly can completely cover any evolution process, and \emph{complex} changes, which consist of a coherent list of interrelated atomic changes about which one has additional knowledge.\LC{TODO: think about phrasing of complex changes a bit more.}% provide more semantically meaningful information.
%\LK{How can we specify ``more semantic information'' better?}

\textcite{khelladi_detecting_2015} list the operations \emph{add}, \emph{delete} and \emph{update} as a complete set of atomic change types, usable on various \metamodel elements.
In combination they are able to describe any difference between two \metamodels as a change sequence.
\textcite{burger_change_2010} go further by defining a change \metamodel, which provides a more formalized complete description of \metamodel changes, and allows \metamodels to be treated as usual models.

In order to define more meaningful change types for the purpose of analyzing \metamodel evolution, \textcite{herrmannsdoerfer_extensive_2011} define a catalogue of 61 \metamodel evolution operators.
%Instead of claiming that this list is complete for all possible \metamodel differences,
They state that they ``consider [the catalogue] complete for practical application'' \autocite{herrmannsdoerfer_extensive_2011}.
Their catalogue is split into primitive operators, such as \emph{Create Class}, and complex operators, such as \emph{Move Property} or \emph{Extract Super Class}.
In contrast to the atomic change types, these operators represent meaningful evolution steps on which further analyses and automated co-evolution support can be built.
In addition to providing a list of \metamodel evolution operators, \textcite{herrmannsdoerfer_extensive_2011} discuss the impact of the described evolution steps on the models adhering to the evolved \metamodels.
For our approach, we will use the most important evolution operators, according to an industry study of \textcite{khelladi_detecting_2015} and explained in \cref{sec:Suggestion:Change}, to characterize the performed evolution steps and analyze their impact on dependent view types.
%\LC{'a subset' sounds weak; can we characterize it better?}

While \textcite{herrmannsdoerfer_extensive_2011} already describe that the complex changes can be built from atomic changes, \textcite{khelladi_change_2018} provide an approach to reconstruct complex changes from a sequence of atomic changes.
Their approach can handle overlapping, incomplete and hidden complex changes, as well as dependencies between them, and can be instantiated for arbitrary sets of complex changes.
Since we do not expect developers to describe the evolution steps performed on a \metamodel manually, we rely on the analysis from \textcite{khelladi_change_2018} to extract the complex changes from a sequence of atomic changes.
The sequence of atomic changes can either be recorded while the \metamodel is edited in a suitable development environment or derived from the difference between the \metamodel before and after the evolution process \autocite{wittler_derivation_2021}.
