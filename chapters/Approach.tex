\section{Suggestions for \viewtype co-evolution} 
\label{sec:Approach}

%\MA{I changed the Section's title from ``Approach'', which IMO does not fit anymore. Feel free though to improve that!}\LC{New title doesn't quite fit, viz.~title of preceding section. Perhaps we change preceding section's title to 'Conceptual Model for Viewtype Co-Evolution'?}

While the \textsf{Suggestion} \metamodel captures how we
conceptualise recommendations for co-evolving \viewtypes, we detail
in this section \textsf{Suggestion}'s ``content'', i.e. which exact \textsf{Recommendation}s
will be issued for each change \textsf{Operator} detected in an evolution
session. This results in Table~\ref{tab:suggestions}, where:
\begin{itemize}
	\item Column \textsl{Operator} lists all \textsf{Operator}s considered in Figure~\ref{fig:Operator}. 
    However, the following \textsf{Operator}s, which are
	associated with severity \textsf{IGNORE}, are excluded from our Table, since 
	they do not result in any \textsf{Recommendation}: atomic operators \textit{Delete Package}, 
	\textit{Make/Drop Class Abstract}, \textit{Make/Drop Attribute Identifier}, 
	\textit{Make/Switch Reference Composite}, \textit{Make/Switch Reference Opposite}
	and complex operator \textit{Pull Feature}. In other words, Table~\ref{tab:suggestions} only
	lists \textsf{Operator}s resulting in actual \textsf{Recommendation}, but all
	others simply do not issue any and thus, correspond to the \textsf{0} multiplicity
	in $\mathsf{Suggestion} \squaredots \mathsf{recommendations}$ in Figure~\ref{fig:Suggestion}.
	
	\item The \textsl{Condition} column corresponds to constraints that are captured
	either by how the \textsf{Operator} is defined (typically, which \textsf{container}
	is involved), or by information captured by traceability \textsf{Link}s (typically,
	an element is referenced in a \viewtype). Sometimes, information coming from \textsf{MM},
	the \metamodel under evolution, may bring other conditions that can be statically
	computed (typically, which sub- or super-classes). 
	
	\item The \textsl{Suggestion} column details, using the actions of Figure~\ref{fig:Recommendation}, which \textsf{Recommendation}s we offer, and which elements in the \viewtype they are applied to.

    \item We divided the table into two parts where the first part lists the \textsf{Primitive} operators and the rest contains the \textsf{Complex} ones. To keep the table relatively less crowded, we do not indicate this in the table.
\end{itemize}

Prior to discussing the suggestions, we briefly focus on the \textsf{Ignore}d operators. The deletion of a package is possible iff it is empty and therefore, this operation can be safely ignored. The \textit{Make Class Abstract} and \textit{Drop Class Abstract} are about making an existing class abstract and vice versa. As these operations only changes the abstraction without modifying the effective list of features, they can be ignored in the context of \viewtype co-evolution. The \textit{Make Attribute Identified} and \textit{Drop Attribute Identifier} operators adds or removes a constraint that ensures the uniqueness of the attribute values. The \textit{Make Reference Composite} and \textit{Drop Reference Composite} adds or removes the containment restriction for an reference. Since, these are instance level constraints, they do not influence the \viewtype and therefore, can be ignored. The \textit{Pull Up} operator can be applied iff the corresponding feature is present in all the child classes. Since the feature is still accessible via inheritance, this does not trigger any suggestion for the related \viewtypes.

Among the primitive operations the creation of various entities are of minor severity as they do not influence the existing relation between the \metamodel and corresponding \viewtypes. For the creation of package, class, data type, attributes, references, literals, and enumeration, we first identify the \viewtypes that reference the corresponding container (i.e., class or enumeration) and suggest the addition of the new entities for these \viewtypes.

Deletion of class, feature, data type, and enumeration can have major effect if these entities are referred from one or more related \viewtypes as the partial representation relation with the corresponding \metamodel no longer holds. Therefore, we suggest the removal of these also from the \viewtypes. The deletion of an opposite reference removes access to the featuers of the referenced entity from the referencing one. The corresponding suggestion is to access these features directly through the referenced entity from the \viewtype. \textit{Merge Literal} removes a literal and replaces its occurrence with another one. Therefore, applying this operation triggers the suggestion to replace the deleted literal with an existing one from the same enumeration. \textit{Rename} and \textit{Change Package} operators generate suggestions to do the same (i.e., renaming and changing package location) on the related \viewtypes.

The \textit{Add Super Type} and \textit{Remove Super Type} operators adds and removes inherited features from an existing class. Accordingly, the corresponding suggestion suggests the addition or removal of inherited features wherever the child class is referenced in the \viewtype. Although the suggestion for the former is of minor severity, the removal of features can break the \viewtype and hence, is classified as major.

The \textit{Move Property} operator triggers the suggestion to update the corresponding location in the \viewtype. Pushing a feature down in the hierarchy removes it from the super class and therefore, it needs to be removed from the \viewtype wherever it is referred using a reference of the super class.

Extracting super class does not influence the effective set of features and therefore, triggers a minor suggestion for replacing some occurrences of the child class reference with the parent. The \textit{Flatten Hierarchy} moves all the features from the parent to the child classes and deletes the parent class. Therefore, this triggers a major suggestion for replacing all the references of the removed super class with the appropriate child class reference. Since the effective features of the child classes remain the same before and after fattening, we offer no suggestion for the child classes. The \textit{Extract Class} operator groups a set of features into a new delegate class and replaces their occurrences with a reference to the delegate class. The \textit{Inline Class} operator is the exact opposite of \textit{Extract Class}. Both triggers a major suggestion of adapting the location of any related feature referenced in the \viewtype.

\begin{table*}[ht!]
\caption{Suggestions per change operator.} 
\label{tab:suggestions}
\centering
\begin{tabular}{|p{.16\linewidth}|p{.30\linewidth}|p{.4\linewidth}|c|}
\hline
Operator & Condition to offer suggestion & Suggestion & Severity \\ \hline \hline

Create Package $(p)$&  
\multirow{8}{*}{\parbox{\linewidth}{The container of $x$ is referred in the \viewtype, where $x \in \{p, c, dt, e, r, or, a, l\}$}} &      
\multirow{8}{*}{$ADD(x)$ in the \viewtypes} &
\multirow{8}{*}{N} \\ \cline{1-1}
Create Class $(c)$&  &    &                   \\ \cline{1-1}
Create Data Type $(dt)$&  &    &                   \\ \cline{1-1}
Create Enum $(e)$&    &  &                   \\ \cline{1-1}

Create Reference ($r$)& &  &                   \\ \cline{1-1}
%    
% \multirow{4}{*}{\parbox{\linewidth}{Containers (i.e., package, class) to which the entity is added are referenced from the \viewtype}} &      
% \multirow{4}{*}{\parbox{\linewidth}{Suggest the addition of the new entities in the \viewtype}} &
% \multirow{4}{*}{N} \\ \cline{1-1}
Create Opposite Ref. ($or$) &   &   &                   \\ \cline{1-1}
Create Attribute ($a$)&  &    &                   \\ \cline{1-1}
Create Literal ($l$)&    &  &                   \\ \hline

% Delete Package  & 
% None & None. & N \\ \hline

Delete Class ($c$)& 
\multirow{4}{*}{\parbox{\linewidth}{$x$ is referenced in the \viewtype, where $x \in \{c, f, dt, e\}$}} &
\multirow{4}{*}{$DEL(x)$ from \viewtype} & \multirow{4}{*}{M}           \\ \cline{1-1}
Delete Feature ($f$) &     & &                   \\ \cline{1-1}
Delete Data Type ($dt$) &    &  &                   \\ \cline{1-1}
Delete Enum ($e$) &   &   &                   \\ \hline

%Make Class Abstract  & \multirow{2}{*}{P} & \multirow{2}{*}{None}     & \multirow{2}{*}{None}     & \multirow{2}{*}{I} \\ \cline{1-1}
%Drop Class Abstract  &  & &  & \\ \hline

Delete Opposite Ref. ($or$) & $f^*$ is referred in \viewtype through referencing type ($r_s$), where $f^*$ is the set of features available via $or$ & $MOVE$ $f^\prime\in f^*$ from accessing through $r_s$ to directly through $or$ & M            \\ \hline

Merge Literal  ($l$)&   $l$ is referenced in \viewtype & $UP$date $l$ with $l^\prime$, where $l$ and $l^\prime$ are contained in the same \textit{Enum} & M            \\ \hline

Rename  &  Old name $s$ is referred in the \viewtype  &  $UP$dating $s$ with $s_c$ where $s_c$ is the current name & M \\ \hline

Change Package ($p_o$) &  $x$ is referred in \viewtype, where $x\in\{p, t, dt, e, c\}$ and package $p_o$ contained $x$ & $MOVE$ of $x$ from $p_o$ to $p_c$, where $p_c$ is the current package & M \\ \hline

Add Super Type ($t$) &  $t_c$ is referred in \viewtype, where $t_c$ is a child type of $t$ & $ADD(f^\prime)$, where $f^*$ is the set of features contained in $t$ and $f^\prime\in f^*$& N  \\ \hline

Remove Super Type ($t$) &  $t_c$ is referred in \viewtype, where $t_c$ is a child type of $t$ & $DEL(f^\prime)$, where $f^*$ is the set of features contained in $t$ and $f^\prime\in f^*$& M \\ \hline
%Make Attr. Identifier   &  \multirow{4}{*}{None}    &  \multirow{4}{*}{None}    & \multirow{4}{*}{I}            \\ \cline{1-1}
%Drop Attr. Identifier  & &      &      &     \\ \cline{1-1}
%Make Ref. Composite  & &      &      &             \\ \cline{1-1}
%Switch Ref. Composite  & &      &      &             \\ \hline
% Make Ref. Opposite  & P&  None\LC{Remove this and next line, see arguments Moussa in running text}    &   None   &             \\ \hline
% Drop Ref. Opposite  & P&   None   &    None  &             \\ \hline
%\hline

Move feature ($f$) &   $f$ is referred in \viewtype  & $MOVE$ $f$ from $c_s$ to $c_d$, where $c_s$ and $c_d$ are previous and current container classes respectively & M \\ \hline

Push feature ($f$)  &  $f$ is referred in \viewtype using a reference of $c_s$, where $c_s$ is a super class from which $f$ was pushed to all its child classes $c_c$ & $MOVE$ $f$ from $c_s$ to $c^\prime$, where $c^\prime\in c_c$   & M \\ \hline
%Pull Property   &  None & None & I \\ \hline

Extract Super Class ($c$) &  $c_c$ is referred in \viewtype, where $c_c$ is the set of child classes of $c$ & $UP$date $c^\prime_c$ with  $c$, where $c^\prime_c\in c_c$ & N  \\ \hline

Flatten Hierarchy (removing super class $c$)  &  $c$ is referenced in \viewtype & $MOVE$ $f^\prime$ from $c$ to $c_c$ for all $f^\prime\in f^*$, where $c_c$ is a child class of $c$ and $f^*$ is the set of features of $c$  &  M           \\ \hline

Extract Class ($c$) &  $f^\prime$ is referred in \viewtype, where $f^*$ is the set of features extracted into delegate class $c_{del}$ and $f^\prime\in f^*$ & 
$MOVE$ $f^\prime$ from $c$ to $c_{del}$ for all $f^\prime$ & M \\ \hline

Inline Class (deleting delegate class $c$) &  $c$ is referenced in \viewtype & $MOVE$ $f^\prime$ from $c$ to $c_{ref}$ where $f^*$ is the set of features from $c$ and $f^\prime\in f^*$ and $c_{ref}$ is a class referencing $c$ & M            \\ \hline

\end{tabular}
\end{table*}