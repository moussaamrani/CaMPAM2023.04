\section{Formalization}
\label{sec:Formalization}
In this chapter we define the terms suggestion and applicable suggestion. We introduce possible ways to refine a suggestion and to add domain specific knowledge.

\begin{definition}[Suggestion]
We define a suggestion $SUG$ as a 4-Tuple of a type of a change on the \metamodel $MMC$, a view generation transformation $VGT$ (which link to the \metamodel and the \viewtype),  constraints $CON$ and a suggestion content $SUC$. Short: $SUG = (MMC, VGT, CON, SUC)$
\end{definition}
Possible $MMC$ are defined in \cref{tab:suggestions}. A suggestion always contains one change. Change sequences that have domain semantics should define a new complex change, because the contained domain knowledge is interesting for other analyses besides suggestions. The view generation operators $VGO$ are defined in \cref{fig:model-query-operators}. They can be combined to a $VGT$. $VGTs$ may form a hierarchy that can be used for the refinement of suggestions. The constraints are defined on either the \metamodel, its instance, the view generation operators or the target \metamodel. Constraints may also form a hierarchy for the refinement of suggestions. The $SUC$ is either a textual description of the suggestion or a formalisation of possible changes or a change on the \viewtype or a combination of all three. Either the textual description or the formalisation and the possible changes are preferable as it enables the methodologist to precisely define the change on the \viewtype \metamodel as well as giving some rationale. The formalization represents the middle way between the directly applicable definition of the changes and the textual description.

An example for a suggestion is \textit{(Create Attribute in metamodel, Select,  "viewtype is identity mapping of metamodel", ("As the viewtype is an identity mapping of the underlying metamodel, we suggest to update the viewtype to also include the new attribute.", Create Attribute in viewtype metamodel))}. A possible refinement could also include the $CON$ \textit{attribute is marked as hidden} which may semantically take precedence over the identity mapping and the suggestion contains the $SUC$ \textit{"As the attribute is marked as hidden, we suggest to not update the viewtype."}. Both $VGT$ and $CON$ enable the methodologist to include semantic domain specific knowledge into the suggestions.