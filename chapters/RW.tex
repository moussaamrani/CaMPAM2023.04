\section{Related Work} \label{sec:RW}

As indicated earlier, the evolution of \metamodels is a well-known phenomenon in the model-driven community. With such evolution, related artefacts must co-evolve to maintain conformance and proper functioning of related software systems, which is becoming increasingly complex and susceptible to frequent changes. Hebig et al.~\autocite{Hebig2017} performed a survey on 31 approaches supporting the co-evolution of \metamodels and their related models, which is known as meta-model model (MM-M) co-evolution. They divided the surveyed approaches into six groups: resolution strategy languages, resolution strategy generation, resolution strategy learning, constrained model search, and identify complex changes. The first group includes seven transform languages for specifying MM-M co-evolution strategies, while the second group contains approaches for the generation of such strategies. The researches included in the third group tries to automate the co-evolution using a set of predefined strategies. The forth group includes one paper by Anguel et al.~\autocite{Anguel2013} that proposes a learning based approach to learn resolution strategies. The fifth group consists of four approaches that apply a constrained-based search of valid model variants considering the original model and the new \metamodel. Finally, the sixth group of works focus on detecting complex \metamodel changes instead of strictly focusing on co-evolution.

Kessentini et al.~\autocite{Kessentini2022} suggested a semi-automated interactive approach for MM-M co-evolution where they generate a set of edit operations on the models based on the \metamodel evolution. Prior to suggesting these operations to the modeller, the approach performs a set of optimisation steps for improving the suggestions. To help the modeller even further in choosing the appropriate set of model edit operations, the approach can cluster the proposed solutions in different categories and predict the impact of each of the operations on the corresponding models.

Furthermore, the co-evolution of \metamodel and related transformations and constraints has been an active research topic in recent years~\autocite{Kusel2015, Khelladi2017, garcia2013}.

Co-evolution has been researched extensively by not only the MDE but also the database management system (DBMS) community. The continuously evolving nature of the modern software systems often demands the modification of the corresponding database schema warranting further co-evolution of entities such as stored data, queries, constraints and so on. Andany et al.~\autocite{andany1991management} in the early 1990s proposed a version management system for database schema. 

\autocite{younggook2005, HICK2006534, curino2010, Curino2008}


% \begin{itemize}
%     \item \metamodel version co-existence (Alexander Egyed)
%     \item MM to MM co-evolution
%     \item database view co-evolution
% \end{itemize}
