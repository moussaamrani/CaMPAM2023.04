\section{Conclusion} \label{sec:Conclusion}

In the context of view-based development, co-evolving \viewtypes together with corresponding \metamodels is crucial for maintaining the soundness and relevancy of the \viewtypes. This is especially important in industrial settings where in-house domain-specific languages evolve rather frequently. In this work, we 
% identified \LK{what does ``identify'' mean here, since we did not come up with them?} and \HM{i meant identified from the literature. but removed it to avoid confusion}
analyzed 34 \metamodel evolution operators---27 primitive and 7 complex ones---and proposed an approach for suggesting possible co-evolution actions on related \viewtypes for each \metamodel evolution operator. We first defined a conceptual model that associated a \metamodel change \textsf{Operator} with a set of \viewtype change \textsf{Recommendation}s, a \textsf{Relation} between the two, and a set of \textsf{Conditions}
% , which can be domain specific\LC{true for current paper?}\HM{not really, removed}, 
ensuring the validity of the recommendation (cf. \cref{sec:Suggestion}). Based on this conceptualization, we presented \viewtype co-evolution \textsf{Suggestion}s for each relevant \metamodel change \textsf{Operator} (cf. \cref{sec:Approach}). Furthermore, we understand that co-evolving \viewtypes and \metamodels require the co-evolution of the corresponding views and models. Therefore, we present and discuss several existing work that focuses on \metamodel/model and related co-evolution scenarios. 

We used the example of the Finite State Machine while presenting our approach for \viewtype/\metamodel co-evolution. Although, this example is adequate for explaining the approach, but it is too simplistic to be a viable method for evaluating its effectiveness.
% this example is sufficient for explaining the approach, it is too simple to serve as a viable means of evaluating the approach. 
Therefore, we aim to extend this work by evaluating the presented approach using one or more \metamodels applied in real-word situations such as the one from \textcite{braun_classification_2014}. Furthermore, we do not tackle the co-evolution of view generation transformations (VGT), which instantiates views based on a \viewtype, with the evolution of corresponding \viewtypes and \metamodels. This can partly be attributed to the present incompleteness of the VGT semantics. However, it remains an open research direction. 

\LC{TODO: Loek got to here.}

\LC{TODO: probably future work: talk about (i) more complex patterns to give better suggestions;
(ii) domain-specific approach / refinements to approach, with illustration on
state charts?}
\TW{I added an example in \cref{sec:Conditions}}

% discuss work, limitations, where it should go

% \st{viewtype transformations}


% Future work - make suggestions applicable 

\begin{comment}
\TW{TODO: remove the formalisation, cut the definition short and try to define the idea more precisely} \begin{definition}[Applicable Suggestion]
We define an applicable suggestion as a 3-Tuple of a suggestion, a transformation of the $VGT$($TVGT$) and a transformation between the $MMC$ and the $SUC$($TSUC$). Short $ASU = (SUG, TVGT, TSUC)$
\end{definition}
\HM{This definition seems unclear to me. I am also not sure of its relevance in the context of this work.} \TW{As discussed in the meeting, this definition is not part of the scope of this paper but future work on how to use our definitions. We may keep it as limitation or move its content to the future work section (a formal definition won't make sense there, but i think we should keep the ideas and present them.)}
Note that the $SUC$ of an $ASU$ has to contain the definition of the change on the \viewtype \metamodel. The $TVGT$ transforms the $VGO$, e.g. by adding a newly created attribute as input. The $TSUC$ transforms the change on the \metamodel in a change on the \viewtype \metamodel. In the example above, the change on the \metamodel will be transformed by changing its target from the \metamodel to the \viewtype \metamodel. A suggestion can have multiple applicable suggestions from which the methodologist can chose the most appropriate one.
\end{comment}
