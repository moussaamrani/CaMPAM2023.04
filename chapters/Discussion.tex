\section{Discussion}
\label{sec:Discussion}

The approach sketched in this paper aims at recommending possible changes 
for realigning \viewtypes, when changes are observed, or detected on a 
\metamodel. Our approach assumes that \metamodel changes are expressed
as a set, or a sequence or change \textsf{Operator}s: we therefore build
on existing approaches for detecting and classifying evolution changes
(cf. \cite{herrmannsdoerfer_extensive_2011,khelladi_detecting_2015} and 
\S \ref{sec:MetaModelEvolution}).

Our work fits into the general topic of \metamodel{} -- \metamodel co-evolution,
but with specific restrictions. First, although \viewtypes are structurally 
plain metamodels, they achieve the specific intent of capturing the relevant
information presented through a system's view. As such, \viewtypes can
be seen as abstractions of the underlying \metamodel{}(s): information from
the original \metamodel{}(s) may be lost (i.e. irrelevant for the view), other
may be presented differently (simplified, aggregated, etc.). Therefore, our
\textsf{Suggestion}s closely follow what an \textsf{Operator} describes on
the evolved \metamodel, but translates this evolution into \emph{all} impacted
elements of \emph{all} \viewtypes, as described in Table \ref{tab:suggestions}.
This is possible thanks to the traceability \textsf{Relation} 
(cf. Fig. \ref{fig:Relation}) that traces which \viewtype elements are impacted
by a change.
Extending our approach to plain \metamodels, instead of \viewtypes, may partially
work depending on the nature of the co-evolved \metamodels depending on how
both \metamodels are related, but they are certainly not directly generalisable.

Second, in our vision of (projective) view-based development, views are generated
from models using View Generation Transformations (VGTs). As we have pointed in our
example, a natural extension point, which may sometimes be equally important,
is to \emph{also} provide \textsf{Suggestion}s at the VGT level to help 
methodologist correct how information is represented (from models to to the views),
but also how edits in views may translate back into the various models specifying
the system. We briefly sketched in our example how simple cases for the first 
direction may be addressed; the reverse direction is more complicated and requires
further research.

Third, evolving a \metamodel and its associated \viewtypes naturally requires to
co-evolve the underlying model(s) and view(s). We would simply reuse existing 
techniques and tools for \metamodel/model co-evolution (as discussed in \S \ref{sec:RW}),
from which, in our approach, views would be automatically generated using VGTs.



\LC{TODO: talk about (i) more complex patterns to give better suggestions;
(ii) domain-specific approach / refinements to approach, with illustration on
state charts }

discuss work, limitations, where it should go

\st{viewtype transformations}



\textbf{Limitations:} \textbf{\textsl{All these items are now coverdd in the discussion above}}
\begin{itemize}
    \item we do not support metamodel/metamodel co-evolution: we’re looking 
		at view-based only, not necessarily V-SUM, so MM-MM co-evolution not a concern
    \item we do not support metamodel/model co-evolution
    \item we only support co-evolution from the metamodel to the viewtype (which is actually not a limitation, since conceptually one can still start at the viewtype)
    \item we only support one metamodel as the source of a viewtype
\end{itemize}