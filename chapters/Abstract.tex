\begin{abstract}
View-based development is a successful approach for the development of complex cyber-physical systems.
It uses views to abstract from the complexity of the system and allows the developers to focus on exactly the necessary information for a certain task.
With projective views, the information shown is derived from underlying models, and changes made to the views are reflected back to the models.
Similar to how models conform to a \metamodel, views conform to a \viewtype, which describes what and how the information is presented.
When the underlying \metamodels need to evolve, e.g., due to new requirements, so do the \viewtypes that rely on them.

In this work, we investigate how to assist the meta-model/view-type co-evolution process by providing suggestions for adapting a \viewtype after a \metamodel change.
To this end, we precisely describe what a suggestion in this context is, and present a list of domain-independent suggestions for the most representative \metamodel evolution steps. 

We describe how to specialize our approach with domain-specific suggestions, and sketch how a tool based on our approach could be implemented---using recently developed techniques for detecting semantically meaningful \metamodel evolution steps.
\MA{Careful that the promise of this last sentence is not in the paper!}

%\TW{\begin{quote}
%    Specifically, in its 2023 edition, the workshop aims to shed light on the strong links between model evolution and sustainability, and in a broader sense, to shed light on the potential of model-centered techniques to drive sustainability initiatives.
%\end{quote} Probably we can say a sentence or two in that direction}
%\LC{Discussed today during call. Seemed to have agreement it's a bit far-fetched, at least, to be more specific than the general 'support for models and model evolution helps for sustainability' claim, and putting a sentence or two in doesn't help for acceptance.}
\end{abstract}

\begin{IEEEkeywords}
View-Based Modelling,
Co-Evolution
\end{IEEEkeywords}
